\textbf{Papers}

\begin{itemize}
\item
  \textbf{Peterson EJ} \& Verstynen T, A way around the
  exploration-exploitation dilemma, bioRxiv 671362v5 (2019).
\item
  \textbf{Peterson EJ} \& Verstynen T, Artificial astrocyte networks,
  \emph{In prep.} (2019).
\item
  Izhikevich L, Gao R, \textbf{Peterson EJ} \& Bradley Voytek, Measuring
  the average power of neural oscillations, bioRxiv 441626 (2018). Under
  review at \emph{J. Neurosci. Methods}.
\item
  \textbf{Peterson EJ}, Müyesser NA, Verstynen T \& Dunovan K, Combining
  Imagination and Heuristics to Learn Strategies that Generalize, Under
  review at \emph{Neurons, Behavior, Data analysis, and Theory}.
\item
  Agarwal A, Kumar AV, Dunovan K, \textbf{Peterson EJ}, VerstynenT \&
  Sycara K, Better safe than sorry: evidence accumulation allows for
  safe reinforcement learning, ArXiv 1809.09147 (2018).
\item
  \textbf{Peterson EJ} \& Voytek B, Homeostatic mechanisms may shape the
  type and duration of oscillatory modulation, bioRxiv 615450v1 (2018).
  Under review at \emph{J Neurophys}.
\item
  \textbf{Peterson EJ} \& Voytek B, Learning with discrete
  representations using continuous chaotic neural populations \emph{In
  prep} (2019).
\item
  \textbf{Peterson EJ} \& Voytek B, Healthy oscillatory coordination is
  bounded by single-unit computation, bioRxiv 309427 (2018).
\item
  Matar Haller{[}1{]}, Thomas Donoghue{[}1{]}, \textbf{Erik
  Peterson}{[}1{]}, Paroma Varma, Priyadarshini Sebastian, Richard Gao,
  Torben Noto, Robert T. Knight, Avgusta Shestyuk, Bradley Voytek,
  Parameterizing neural power spectra, bioRxiv 29985 (2018). {[}1{]}:
  Co-first. Under review at \emph{Nature Neuroscience}.
\item
  \textbf{Peterson EJ} \& Voytek B, Alpha rhythmically alters gain by
  modulating excitatory-inhibitory background activity, bioRxiv 185074v2
  (2017).
\item
  Gao RD, \textbf{Peterson EJ}, Voytek B, Inferring synaptic
  excitation/inhibition balance from field potentials, Neuroimage
  Sep;158:70-78 (2017).
\item
  \textbf{Peterson EJ}, Burke QR, Campbell AM, Belger A, Voytek B, 1/f
  neural noise is a better predictor of schizophrenia than neural
  oscillations, bioRxiv 113449v4 (2017)
\item
  Cole SR, \textbf{Peterson EJ}, van der Meij R, Hemptinne C, Starr PA,
  \& Voytek B, Nonsinusoidal oscillations underlie pathological
  phase-amplitude coupling in the motor cortex in Parkinson's disease, J
  Neurosci 37(18) (2017)
\item
  \textbf{Peterson EJ} and Voytek B, Balanced oscillatory coupling
  improves information flow, bioRxiv 030304v2 (2016).
\item
  \textbf{Peterson EJ} and Seger CA, In model-based fMRI significant is
  less than specific., bioArxiv 429621 (2017).
\item
  \textbf{Peterson EJ}, Seger CA and Anderson CA, Many Hats: Changes in
  the Striatal Bold Signal Across Stimulus, Preparation, Response and
  Feedback, Journal of Neurophysiology 110(7) 1689-1702 (2013).
\item
  Seger CA and \textbf{Peterson EJ}, Categorization = Decision Making -
  Generalization, Neurosci Biobehav Rev 37(7) pp1187-1200 (2013).
\item
  Seger CA, Dennison CM, Lopez-Paniagua DL, \textbf{Peterson EJ}, and
  Roark AA, Dissociating Hippocampal and Basal Ganglia Contributions to
  Category Learning Using Stimulus Novelty and Subjective Judgments,
  Neuroimage 55(4), pp1739-53 (2011).
\item
  Seger CA, \textbf{Peterson EJ}, Cincotta C, Lopez-Paniagua DL and
  Anderson C, Dissociating the Contributions of Independent
  Corticostriatal Systems to Visual Categorization Learning Through the
  Use of Reinforcement Learning Modeling and Granger Causality Modeling,
  NeuroImage 50(2) pp644-656 (2010).
\item
  Bedoukian MA, Whitesell J, \textbf{Peterson EJ}, Clay C and Partin KM,
  The Stargazin C Terminus Encodes an Intrinsic and Transferable
  Membrane Sorting Signal, J. Biol. Chem., 283(3), pp1597-1600 (2008).
\item
  Johansson HE, Johansson MK, Wong AC, Armstrong ES, \textbf{Peterson
  EJ}, Grant RE, Roy MA, Reddington MV and Cook RM, BTI1, an
  Azoreductase with pH Dependent Substrate Specificity, Appl Environ
  Microbiol Jun;77(12):4223-5 (2012).
\item
  Cheung CL, Rubinstein AI, \textbf{Peterson EJ}, Chatterji A,
  Sabirianov RF, Mei W, Lin T, Johnson JE and DeYoreo JJ, Steric and
  Electrostatic Complementarity in the Assembly of Two-Dimensional Virus
  Arrays, Langmuir 26 (5) pp3498--3505 (2010).
\end{itemize}

\textbf{Posters}

\begin{itemize}
\item
  \textbf{Peterson EJ} \& Verstynen T, A way around the
  exploration-exploitation dilemma, presented at Conference on Cognitive
  Computational Neuroscience (CCN), Berlin Germany, 2019.
\item
  \textbf{Peterson EJ} \& Verstynen T, Artificial astrocyte networks,
  presented at The Bernstien Conference, Berlin Germany, 2019.
\item
  \textbf{Peterson EJ} \& Voytek B, Homeostasis and oscillatory
  modulation, presented at Society for Neuroscience (SFN), San Diego CA,
  2018.
\item
  \textbf{Peterson EJ}, Müyesser NA, Dunovan K \& Verstynen T, Combining
  heuristics with counterfactual play in reinforcement learning,
  presented at Conference on Cognitive Computational Neuroscience (CCN),
  Philadelphia PA, 2018
\item
  \textbf{Peterson EJ} \& Voytek B, The tradeoff between oscillatory
  coordination and neural computation, presented at Society for
  Neuroscience (SFN), Washington DC, 2017.
\item
  \textbf{Peterson EJ} \& Voytek B, Gain control across cortical layers
  can be mediated by balanced oscillatory coupling, presented at Society
  for Neuroscience (SFN), San Diego, CA 2016.
\item
  Haxby S \& \textbf{Peterson EJ}, Learning with discrete
  representations using continuous chaotic neural populations, presented
  at Society for Neuroscience (SFN), San Diego, CA 2016.
\item
  Gao R, \textbf{Peterson EJ}, \& Voytek V, Spiking correlates and
  temporal variability of oscillatory frequency modulation, presented at
  Society for Neuroscience (SFN), San Diego, CA 2016.
\item
  Rosen BQ, \textbf{Peterson EJ}, Campbell AM, Belger A \& Voytek B,
  Spectral 1/f noise differences account for apparent oscillatory
  band-specific effects in Schizophrenia, presented at Society for
  Neuroscience (SFN), San Diego, CA 2016.
\item
  L. Izhikevich L, \textbf{Peterson EJ} and Voytek B, Neural oscillatory
  power is not Gaussian distributed across time, presented at Society
  for Neuroscience (SFN), San Diego, CA 2016.
\item
  \textbf{Peterson EJ} \& Wheeler MW, The diversity of distributed
  decisions, presented at Society for Neuroscience (SFN), San Diego, CA
  2015.
\item
  \textbf{Peterson EJ} \& Voytek B. Spike-field coupling does not imply
  spike-spike coupling, presented at Society for Neuroscience (SFN), San
  Diego, CA 2015.
\item
  Noto T, Gao R, \textbf{Peterson EJ}, Voytek B. Neural network
  properties can be inferred from electrophysiological power spectral
  geometry, presented at Society for Neuroscience (SFN), San Diego, CA
  2015.
\item
  Cole SR, \textbf{Peterson EJ}, de Hemptinne C, Starr PA, Voytek B.
  Deep brain stimulation increases motor cortical 1/f noise and
  decouples high gamma amplitude from beta phase, presented at Society
  for Neuroscience (SFN), San Diego, CA 2015.
\item
  \textbf{Peterson EJ} \& Seger CS, A precise problem in model-based
  fMRI?, presented at Cognitive Neuroscience Society Meeting (CNS), San
  Francisco, CA, May 2013.
\item
  \textbf{Peterson EJ} \& Seger CS, Evidence for generalizable reward
  representations in the basal ganglia examined using fMRI and
  reinforcement learning, International Meeting of the Basal Ganglia
  Society 11, Eilat, Israel, March 2013.
\item
  \textbf{Peterson EJ} \& Wheeler M, Looking everywhere for the right
  model of perceptual decision making, Computational Neuroscience Poster
  Session, Center for the Neural Basis of Cognition, Pittsburgh, PA,
  January 2013.
\item
  \textbf{Peterson EJ} \& Seger CA, Many Hats: Using fMRI to
  Characterize the Roles and Reward Sensitivity of the Striatum Across
  Stimulus, Response and Feedback., International Meeting of the Basal
  Ganglia Society 10, Long Branch, NJ, 2010.
\item
  \textbf{Peterson EJ} and Seger, CA, Reward-level dependent activity
  proceeding and following response selection: an fMRI study, presented
  at SFN2009, Chicago, IL, Fall 2009.
\item
  \textbf{Peterson EJ} and Seger, CA, To Do the Right Thing: Temporal
  Difference Learning As Tool to Dissect the Role of Feedback in the
  Striatum, presented at Cognitive Neuroscience Society Meeting (CNS),
  San Francisco, CA, May 2007.
\item
  Wong MK, Armstrong ES, \textbf{Peterson EJ}, Grant RE, Cook RM, and
  Johnanssen HJ, The BIT1 Azoredustase Colormatric and Fluormetric
  Reporter System, presented at Experimental Biology 2009, New Orleans,
  April 2009.
\item
  Sowers BA, \textbf{Peterson EJ}, Grant RE, Lin WY, Dick DJ and Cook
  RM, Optimization of Probe Performance in Real-Time PCR through an
  Understanding of Synthesis Impurities, presented at Quantitative PCR,
  San Diego (CA) March, 2005.
\item
  \textbf{Peterson EJ}, Weeks BL, De Yoreo JJ, and Schwartz PV, Effect
  of Environmental Conditions on Dip Pen Nanolithography of
  Mercaptohexadecanoic Acid, J. Phys. Chem B (2004), 108 (39),
  pp15206-15210.
\end{itemize}

\textbf{Theses}

\begin{itemize}
\tightlist
\item
  EJ. Peterson,
  \href{https://github.com/parenthetical-e/dissertation}{Rewards are
  Categories?}, PhD Dissertation (2012).
\end{itemize}
